\title{6.867 Homework 2}
\date{\today}

\documentclass[12pt]{article}

\usepackage{graphicx}

\begin{document}
\maketitle

%\begin{abstract}
%Foo bar baz
%\end{abstract}

\section{Logistic Regression}
Behavior at lambda=0: overfitted, weights too large

Behavior as lambda increases, weights shrink in magnitude, but accuracy drops (?? try more values perhaps)

Comment on non-separable data in general.

Second-order basis function. Discuss performance improvements.

\section{SVM implementation}

Run svm on data, report results and discuss

\section{SVM interpretation}

1) Example problem
a) why no change for c > 1
b) manually find soln. is it unique?

2) Try C = [10**(i-2) for i in range(5)]
a) What happens to 1/magnitude(w) as C increases? Will this always happen?
b) What happens to the number of support vectors as C increases?
c) Why is maximizing geometric margin on training set not appropriate criterion for picking C. Alternative?

3) How does xi relate to distance of support vector from decision boundary? (Bishop)

4) Optimal slack loss function? Additional constraints?

\section{Kernel SVM}

Test second order polynomial kernel and Gaussian kernel. Show results, explain, esp. mistakes for several values of C and the Gaussian kernel variance 1/beta. Compare to results from logreg.

\end{document}
